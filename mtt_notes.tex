\documentclass[a4paper,10pt]{scrreprt}
\usepackage[top=2cm,bottom=4cm]{geometry}
\usepackage[english]{babel}
\usepackage[T1]{fontenc}  			% make sure to install cm-super package for improved PDF quality
\usepackage[utf8]{inputenc}
\usepackage{lmodern}
%\usepackage[cm]{fullpage}			% Fullpage for now
\usepackage{graphicx}
\usepackage[dvipsnames]{xcolor}
\usepackage[font=small]{caption}
\usepackage{subcaption}
\usepackage{microtype}			 	% microtypographical fine-tunning
\usepackage{setspace} 				% Provides support for setting the spacing between lines in a document.
%\usepackage{tocbibind}				% Add bibliography/index/contents to Table of Contents.
\usepackage{mathtools}				% enhances amsmath (no need to load amsmath)
\usepackage{amssymb}
\usepackage{amsfonts}
\usepackage{amsthm}
\usepackage{bm} 					% bold math symbols including greek letters
\usepackage{booktabs}				% professional tables (w/o vertical bars)
\usepackage{array}					% 
\usepackage{dcolumn} 				% align to decimal point in tables
\usepackage[shortcuts]{extdash} 	% hyphenation of dashed words
%\usepackage[algoruled]{algorithm2e}
%\newcommand\mycommfont[1]{\small\ttfamily #1}
%\SetCommentSty{mycommfont}
\usepackage{cleveref}				% clever environment referencing
\crefname{figure}{Fig.}{Figs.}
\crefname{algorithm}{Alg.}{Algs.}
\usepackage{siunitx}				% typesetting of SI units
%\usepackage{paralist}		     	% compact lists with more options
%\usepackage[]{todonotes}			% TODO notes
%\presetkeys{todonotes}{backgroundcolor=orange!50}{}
% Not sure which to use for bibliography
%\usepackage{cite}
%\usepackage{citesort}
\usepackage[square, sort, numbers, authoryear]{natbib}
\usepackage{physics}

% ======================== MACROS ==========================
% TODO and other notes
\DeclareDocumentCommand \note 		{ m }{ \textcolor{OrangeRed}{\ \\ \emph{\textbf{(NOTE)}} #1}\ \\ }

% MATH OPERATORS ===============================================================
\DeclareMathOperator*{\argmin}{ \arg\min }
\DeclareMathOperator{\atantwo}{\mathrm{atan2}}
\DeclareMathOperator{\sign}{\mathrm{sign}}
\DeclareMathOperator{\vect}{vec}
\DeclareMathOperator{\pdeg}{pdeg}

% BASIC MACROS =================================================================
% Bold vector and matrix
\DeclareDocumentCommand \vb{ s m } {
	\IfBooleanTF #1 {
		\bm{\lowercase{#2}}
	}{
		\bm{\mathbf{\lowercase{#2}}}
	}
}
\DeclareDocumentCommand \mb { s m } {
	\IfBooleanTF #1 {
		\bm{\uppercase{#2}}
	}{
		\bm{\mathbf{\uppercase{#2}}}
	}
}
% Matrix inverse and transpose
\DeclareDocumentCommand \I		{}{ ^{-1} }
\DeclareDocumentCommand \T		{}{ ^{\top} }
% Diagonal matrix
\DeclareDocumentCommand \diag 	{ l m }{ \mathrm{diag}\pqty#1{\, #2 \,} }
% Unit matrix
\DeclareDocumentCommand \eye 	{ }{ \mb{I} }
% zero vector
\DeclareDocumentCommand \vzero  { }{ \vb{0} }
% elementary vector
\DeclareDocumentCommand \eVec  	{ }{ \vb{e} }
% unit vector
\DeclareDocumentCommand \uVec  	{ }{ \vb{1} }
% column operator
\DeclareDocumentCommand \col  	{ m m }{ \bqty{#1}_{*#2} }
% Integral sign and differential
\DeclareDocumentCommand \integ 	{ O{} O{} }{ \int\limits_{#1}^{#2}\! }
\DeclareDocumentCommand \iinteg { O{} O{} }{ \iint\limits_{#1}^{#2}\! }
\DeclareDocumentCommand \iiinteg{ O{} O{} }{ \iiint\limits_{#1}^{#2}\! }
\DeclareDocumentCommand \dx 	{ O{x} }{ \,\mathrm{d}#1 }
% Expectation, variance and covariance
\DeclareDocumentCommand \E 		{ O{} l m }{ \mathbb{E}_{#1}\bqty#2{ #3 } }
\DeclareDocumentCommand \V 		{ O{} l m }{ \mathbb{V}_{#1}\bqty#2{ #3 } }
\DeclareDocumentCommand \Cov 	{ O{} l m }{ \mathbb{C}_{#1}\bqty#2{ #3 } }
% Kullback-Leibler divergence
\DeclareDocumentCommand \KL 	{ m m }{ \mathbb{KL} \bqty{\left. #1 \,\middle\|\, #2 \right.} }
% Density of Gaussian, Student's t, Gamma and Inverse Gamma variable
\DeclareDocumentCommand \N 		{ s O{\vb{x}} O{\vb{0}} O{\eye} }
{
	\IfBooleanTF{#1}{
		#2\sim\mathrm{N}\pqty{ #3,\, #4 }
	}{
		\mathrm{N}\pqty{\left. #2 \,\middle\vert\, #3,\, #4 \right.}
	}
}
\DeclareDocumentCommand \pdfc	{ O{p} m m }
{
	#1\pqty{\left. #2 \,\middle\vert\, #3 \right.} 
}

% STATE ESTIMATION =============================================================
% time index and final time index
\DeclareDocumentCommand	\tind		{ }{ k }
\DeclareDocumentCommand	\Tind		{ }{ K }
% general nonlinear fcn, system dynamics and measurement fcn
\DeclareDocumentCommand \nlFcn 		{ s }{ \IfBooleanTF{#1}{g}{\vb{g}} }
\DeclareDocumentCommand \stFcn		{  }{ \vb{f} }
\DeclareDocumentCommand \msFcn		{  }{ \vb{h} }
% Jacobian
\DeclareDocumentCommand \jacb 		{ m }{ \mb{g}\pqty{#1} }
% state and measurement variables
\DeclareDocumentCommand \stVar 		{ s }{ \IfBooleanTF{#1}{x}{\vb{x}} }
\DeclareDocumentCommand \msVar		{ s }{ \IfBooleanTF{#1}{z}{\vb{z}} }
% state and measurement noise
\DeclareDocumentCommand \stnVar		{ s }{ \IfBooleanTF{#1}{q}{\vb{q}} }  % TODO rename to \snVar, \stnVar
\DeclareDocumentCommand \msnVar		{ s }{ \IfBooleanTF{#1}{r}{\vb{r}} }
% state and measurement noise covariances
\DeclareDocumentCommand \stnCov		{ s }{ \IfBooleanTF{#1}{\sigma^2_q}{\mb{Q}} }
\DeclareDocumentCommand \msnCov		{ s }{ \IfBooleanTF{#1}{\sigma^2_r}{\mb{R}} }
% state and measurement mean, covariance and cross-covariance
\DeclareDocumentCommand \stMean		{   }{ \vb{m}^x }
\DeclareDocumentCommand \msMean		{   }{ \vb{m}^z }
\DeclareDocumentCommand \stCov		{   }{ \mb{P}^x }
\DeclareDocumentCommand \stMsCov	{   }{ \mb{P}^{xz} }
\DeclareDocumentCommand \msStCov	{   }{ \mb{P}^{zx} }
\DeclareDocumentCommand \msCov		{   }{ \mb{P}^z }
% multi-state, multi-measurement
\DeclareDocumentCommand \mstVar 	{   }{X}
\DeclareDocumentCommand \mmsVar 	{   }{Z}
% filtered state mean and covariance
\DeclareDocumentCommand \filtMean 	{ O{\tind} }{ \stMean_{#1|#1} }
\DeclareDocumentCommand \filtCov 	{ O{\tind} }{ \stCov_{#1|#1} }
% Kalman gain
\DeclareDocumentCommand \kGain		{  }{ \mb{K} }
% input and output dimension
\DeclareDocumentCommand \inDim 		{  }{ D }
\DeclareDocumentCommand \outDim 	{  }{ E }
% state and measurement dimension
\DeclareDocumentCommand \stDim 		{  }{ {d_x} }
\DeclareDocumentCommand \msDim 		{  }{ {d_z} }
\DeclareDocumentCommand \stnDim 	{  }{ {d_q} }
\DeclareDocumentCommand \msnDim		{  }{ {d_r} }
% general decoupled input variable, input variable, output variable
\DeclareDocumentCommand \inVar 		{ s }{ \IfBooleanTF{#1}{{x}}{{\vb{x}}} }
\DeclareDocumentCommand \inVarU		{ s }{ \IfBooleanTF{#1}{{\xi}}{{\vb{\xi}}} }
\DeclareDocumentCommand \inVarC		{ s }{ \IfBooleanTF{#1}{{\eta}}{{\vb{\eta}}} }
\DeclareDocumentCommand \outVar 	{ s }{ \IfBooleanTF{#1}{y}{\vb{y}} }
% input mean, covariance, cov. square-root
\DeclareDocumentCommand \inMean 	{  }{ \vb{m} }
\DeclareDocumentCommand \inCov 		{  }{ \mb{P} }
\DeclareDocumentCommand \inCovFct	{  }{ \mb{L} }
% output mean, covariance and cross-covariance
\DeclareDocumentCommand \outMean 	{  }{ \vb{\mu} }
\DeclareDocumentCommand \outCov 	{  }{ \mb{\Pi} }
\DeclareDocumentCommand \inoutCov 	{  }{ \mb{C} }
% approximate output mean, covariance and cross-covariance
\DeclareDocumentCommand \outMeanApp	{  }{ \vb{\mu}_{\mathrm{A}} }
\DeclareDocumentCommand \outCovApp 	{  }{ \mb{\Pi}_{\mathrm{A}} }
\DeclareDocumentCommand \inoutCovApp{  }{ \mb{C}_{\mathrm{A}} }
% set of reals, naturals, dataset, state space, measurement space
\DeclareDocumentCommand \R 			{  }{ \mathbb{R} }
\DeclareDocumentCommand \Nat		{  }{ \mathbb{N} }
\DeclareDocumentCommand \D 			{  }{ \mathcal{D} }
\DeclareDocumentCommand \Cinf		{  }{ \mathcal{C}^{\infty} }
\DeclareDocumentCommand \stSp		{  }{ \mathcal{X} }
\DeclareDocumentCommand \msSp		{  }{ \mathcal{Z} }
% degrees of freedom parameter: arbitrary, TPQ related and a Student update related
\DeclareDocumentCommand \inDof 		{  }{ \nu }
\DeclareDocumentCommand \nlfDof 	{  }{ \nu_{\nlFcn*} }
\DeclareDocumentCommand \stDof 		{  }{ \nu^x }
\DeclareDocumentCommand \msNoiseDof	{  }{ \nu^r }
\DeclareDocumentCommand \stNoiseDof	{  }{ \nu^q }
\DeclareDocumentCommand \filtDof 	{ O{\tind} }{ \stDof_{#1|#1} }
\DeclareDocumentCommand \fixDof 	{ O{\tind} }{ \nu^* }

% RANDOM FINITE SET FILTERING ===================================
% set differential
\DeclareDocumentCommand \dsx	{ O{X} }{ \mathrm{\delta}#1 }
% multi-targte state related quantities
\DeclareDocumentCommand \stSet	{ }{ X }
\DeclareDocumentCommand \stRFS	{ }{ X }
% set of measurements
\DeclareDocumentCommand \msSet	{ }{ Z }
\DeclareDocumentCommand \msRFS	{ }{ Z }



% ======================== THEOREMS ========================
\theoremstyle{theorem}
\newtheorem{thm}{Theorem}
\newtheorem{lem}{Lemma}
\newtheorem{cor}{Corollary}
\theoremstyle{definition}
\newtheorem{defn}{Definition}
\newtheorem{alg}{Algorithm}
% ========================================================


% Title Page
\title{Notes on Multi-target Tracking}
\author{Jakub Prüher}


\begin{document}
%\maketitle

%\begin{abstract}
%	This report summarizes my understanding of the multi-target tracking (MTT) problems and the most representative algorithms. 
%	First, I cover prerequisites necessary for understanding multi-target tracking such as basics of single-target tracking (Bayesian filtering), measurement formation and data association.
%	The aim is to describe the most representative MTTs, such as joint probabilistic data association filter (JPDAF), multiple hypothesis tracking (MHT) and finally, my favourite, random finite set (RFS) approaches.
%\end{abstract}


%\chapter{Prerequisites}\label{ch:prerequisites}
%\section{Single-Target Tracking}\label{sec:single-target_tracking}
%\section{Single-Target Tracking in Clutter}\label{sec:single-target_tracking_in_clutter}
%\section{Data Association}\label{sec:data_association}
%Global nearest neighbor (GNN)\\
%All-neighbors data association - broader concept, PDA is a special case\\
%Gating - 
%\subsection{Probabilistic Data Association}
%\chapter{Joint Probabilistic Data Association Filter}\label{ch:jpda_filter}
%\chapter{Multiple Hypothesis Tracking}\label{ch:multiple_hypothesis_tracking}


\chapter{Random Finite Sets}\label{ch:random_finite_sets}
% NOTE: the convention that upper-case letters denote number of elements and lower-case counterpart indexes the elements needs to be abandoned, because upper-case is now reserved for RVs and lower-case for RV realizations.
% Thus, N = discrete random variable, n = realization of N (= number of elements in RFS X)

\begin{defn}[\emph{Set Integral}]
	 of a set function \( \pi \) is defined as
	\begin{equation}\label{eq:set_integral}
		\int_S \pi(X) \dsx[X] \triangleq \sum_{k=0}^{\infty}\frac{1}{k!}\int_{S^k} \pi(\Bqty{\inVar_1, \inVar_2, \ldots, \inVar_k}) \dx[(\inVar_1, \inVar_2, \ldots, \inVar_k)]
	\end{equation}
	with the convention \( \int_{S^0} \pi(\emptyset) \dsx[\emptyset] = \pi(\emptyset) \).
\end{defn}

\begin{defn}[\emph{FISST Density}]
	 is a set function \( \pi(X) \) such that 
	\begin{equation}\label{eq:fiist_density}
		\Pr(S \subseteq \mathcal{X}) = \int_S \pi(X) \dsx[X]
	\end{equation}
\end{defn}
FISST density \emph{is not} a probability density.

\begin{defn}[\emph{Cardinality Distribution}]
	Cardinality distribution of a RFS \( X \) is given by
	\begin{equation}\label{key}
		\rho(n) = \Pr(|X|=n) = \frac{1}{n!}\int \pi(\Bqty{\stVar_1,\, \ldots,\, \stVar_n}) \dx[\stVar_1]\cdot\ldots\cdot\dx[\stVar_n]
	\end{equation}
	It's nothing but a probability mass function (PMF) of a discrete random variable \( N \sim \rho(n) \).
\end{defn}

\begin{defn}[\emph{I.I.D. Cluster RFS}]
	RFS \( X \) of cardinality \( n \) with FISST density
	\begin{equation}\label{key}
		\pi(X) = n! \rho(n) \prod_{i=1}^{n} p(\stVar_i).
	\end{equation}
	The RFS contains elements \( \stVar_1, \ldots, \stVar_n \) that are i.i.d. according to a probability density \( p(\stVar): \stSp\to\R \).
\end{defn}

\begin{defn}[\emph{Poisson RFS}]
	I.I.D cluster with Poisson cardinality density, that is
	\begin{equation}\label{key}
		\rho(n \mid \lambda) = e^{-\lambda} \frac{\lambda^n}{n!}
	\end{equation}
	where \( \lambda \) is the expected number of objects.
\end{defn}

\begin{defn}[\emph{Bernoulli RFS}]
\begin{equation}\label{eq:bernoulli_rfs}
	\pi(X) = 
	\begin{cases}
	1 - r, 			& X = \emptyset \\
	r p(\stVar), 	& X = \Bqty{\stVar}
	\end{cases}
\end{equation}
\end{defn}

\begin{defn}[\emph{Multi-Bernoulli RFS}]
	Union of independent Bernoulli RFSs.
\end{defn}

\begin{defn}[\emph{Probability Hypothesis Density}]
	For a RFS \( X \) and , the intensity function is defined as
	\begin{equation}\label{key}
		v(\stVar) = \int \pi(\Bqty{\stVar} \cup X) \dsx[X]
	\end{equation}
	Integrating intensity over a (deterministic) subset of state space \( S \subseteq \mathcal{X} \) yields the expected number of elements (expected cardinality) in \( S \)
	\begin{equation}\label{key}
		\E{|X \cap S|} = \int_S v(\stVar) \dx[\stVar]
	\end{equation}
\end{defn}
Probability hypothesis density (PHD) is also known as intensity function.

\begin{figure}[h]
	\centering
	\includegraphics[scale=0.22]{./img/intensity_function}
	\caption{Intensity function, a.k.a. probability hypothesis density (PHD).}
\end{figure}




\section{Multi-target State-Space Models}

\begin{figure}[h]
	\centering
	\includegraphics[scale=0.25]{./img/multi-target_state_space_model}
	\caption{Multi-target state space model.}
\end{figure}

\subsection{Motion model}
Given a multi-target state \( \stRFS_{\tind-1} \) at time \( \tind-1 \), the multi-target state \( \stRFS_\tind \) at time \( \tind \) is given by the union of the surviving targets, the spawned targets and the spontaneous births
\begin{equation}\label{eq:multitarget_dynamics}
	\stRFS_\tind = \bqty{ \bigcup_{\vb{\zeta} \in \stRFS_{\tind-1}} S_{\tind|\tind-1}(\vb{\zeta}) } \cup \bqty{ \bigcup_{\vb{\zeta} \in \stRFS_{\tind-1}} B_{\tind|\tind-1}(\vb{\zeta}) } \cup \Gamma_\tind
\end{equation}
where
\begin{table}[h]
\centering
\begin{tabular}{@{} l l @{}}
	\toprule
	\( S_{\tind|\tind-1}(\vb{\zeta}) \)		& RFS modeling the behaviour of a target at time step \( \tind \) given a state at time \( \tind-1 \); \\
											& takes on the value \( \Bqty{\inVar_\tind} \) when target survives, or \( \emptyset \) when the target dies. \\
	\( B_{\tind|\tind-1}(\vb{\zeta}) \)		& RFS of targets spawned at time \( \tind \) from a target with previous state \( \vb{\zeta} \) \\
	\( \Gamma_\tind \)						& RFS of spontaneous birth at time \( \tind \) \\
	\bottomrule
\end{tabular}
\end{table}
The actual forms of \( B_{\tind|\tind-1}(\cdot) \) and \( \Gamma_\tind \) are problem dependent.
\begin{figure}[h]
	\centering
	\includegraphics[scale=0.2]{./img/multi-target_motion_model}
	\caption{Multi-target motion model.}
\end{figure}

\subsection{Observation model}
\begin{equation}\label{eq:multitarget_observation}
	\msRFS_\tind = \bqty{ \bigcup_{\vb{\zeta} \in \stRFS_{\tind-1}} \Theta_{\tind}(\inVar) } \cup K_\tind
\end{equation}
where
\begin{table}[h]
	\centering
	\begin{tabular}{@{} l l @{}}
		\toprule
		\( \Theta_{\tind}(\inVar) \)	& can take on \( \Bqty{\msVar_\tind} \) when the target is detected, or \( \emptyset \) when the target is not detected, \\
		\( K_\tind \)					& clutter (false measurements/alarms) \\
		\bottomrule
	\end{tabular}
\end{table}
The actual form of \( K_\tind \) is problem dependent.
\begin{figure}[h]
	\centering
	\includegraphics[scale=0.22]{./img/multi-target_observation_model}
	\caption{Multi-target observation model.}
\end{figure}




\section{Probability Hypothesis Density Filter}
Propagates only the first moment of the RFS.

\begin{table}[h]
\centering
\begin{tabular}{@{} l l @{}}
	\toprule
	\( \gamma_\tind(\cdot) \)								& intensity of the birth RFS \( \Gamma_\tind \) at time \( \tind \) \\
	\( \beta_{\tind|\tind-1}(\cdot \mid \vb{\zeta}) \) 		& intensity of the RFS \( B_\tind(\vb{\zeta}) \) spawned at time \( \tind \) by a target with previous state \( \vb{\zeta} \) \\
	\( \kappa_\tind(\cdot) \)								& intensity of the clutter RFS \( K_\tind \) at time \( \tind \) \\
	\( p_{S,k}(\vb{\zeta}) \)								& probability of target survival at time \( \tind \) given that its previous state is \( \vb{\zeta} \) \\
	\( p_{D,k}(\inVar) \)									& probability of target detection at time \( \tind \) given a state \( \inVar \) \\
	\bottomrule
\end{tabular}
\end{table}

\ \\
\textbf{Assumptions}\\
\begin{description}
	\item[A1] Each target evolves and generates observations independently of one another,
	\item[A2] Clutter is Poisson and independent of target-originated measurements,
	\item[A3] The predicted multi-target RFS governed by \( p_{\tind|\tind-1} \) is Poisson.
\end{description}


Prediction
\begin{equation}\label{eq:phd_filter_prediction}
	v_{\tind|\tind-1}(\inVar) 
	= \integ p_{S,\tind}(\vb{\zeta}) \stFcn_{\tind|\tind-1}(\inVar \mid \vb{\zeta}) v_{\tind-1}(\vb{\zeta}) \dx[\vb{\zeta}]
	+ \integ \beta_{\tind|\tind-1}(\inVar \mid \vb{\zeta}) v_{\tind-1}(\vb{\zeta}) \dx[\vb{\zeta}] + \gamma_\tind(\inVar)
\end{equation}

measurement update
\begin{equation}\label{eq:phd_filter_update}
	v_\tind(\inVar) = (1 - p_{D,\tind}(\inVar))v_{\tind|\tind-1}(\inVar) + \sum\limits_{\msVar \in \msSet_\tind} \frac{p_{D,\tind}(\inVar)\msFcn_\tind(\msVar \mid \inVar) v_{\tind|\tind-1}(\inVar)}{\kappa_\tind(\msVar) + \integ p_{D,\tind}(\inVarU) \msFcn_\tind(\msVar\mid\inVarU) v_{\tind|\tind-1}(\inVarU) \dx[\inVarU]} 
\end{equation}



\bibliographystyle{plainnat}
\bibliography{refdb}
\end{document}          
