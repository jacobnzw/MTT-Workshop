\documentclass[a4paper,10pt]{scrreprt}
\usepackage[top=2cm,bottom=4cm]{geometry}
\usepackage[english]{babel}
\usepackage[T1]{fontenc}  			% make sure to install cm-super package for improved PDF quality
\usepackage[utf8]{inputenc}
\usepackage{lmodern}
%\usepackage[cm]{fullpage}			% Fullpage for now
\usepackage{graphicx}
\usepackage[dvipsnames]{xcolor}
\usepackage[font=small]{caption}
\usepackage{subcaption}
\usepackage{microtype}			 	% microtypographical fine-tunning
\usepackage{setspace} 				% Provides support for setting the spacing between lines in a document.
%\usepackage{tocbibind}				% Add bibliography/index/contents to Table of Contents.
\usepackage{mathtools}				% enhances amsmath (no need to load amsmath)
\usepackage{amssymb}
\usepackage{amsfonts}
\usepackage{amsthm}
\usepackage{bm} 					% bold math symbols including greek letters
\usepackage{booktabs}				% professional tables (w/o vertical bars)
\usepackage{array}					% 
\usepackage{dcolumn} 				% align to decimal point in tables
\usepackage[shortcuts]{extdash} 	% hyphenation of dashed words
%\usepackage[algoruled]{algorithm2e}
%\newcommand\mycommfont[1]{\small\ttfamily #1}
%\SetCommentSty{mycommfont}
\usepackage{cleveref}				% clever environment referencing
\crefname{figure}{Fig.}{Figs.}
\crefname{algorithm}{Alg.}{Algs.}
\usepackage{siunitx}				% typesetting of SI units
%\usepackage{paralist}		     	% compact lists with more options
%\usepackage[]{todonotes}			% TODO notes
%\presetkeys{todonotes}{backgroundcolor=orange!50}{}
% Not sure which to use for bibliography
%\usepackage{cite}
%\usepackage{citesort}
\usepackage[square, sort, numbers, authoryear]{natbib}
\usepackage{sty/macros}


% ======================== THEOREMS ========================
\theoremstyle{theorem}
\newtheorem{thm}{Theorem}
\newtheorem{lem}{Lemma}
\newtheorem{cor}{Corollary}
\theoremstyle{definition}
\newtheorem{defn}{Definition}
\newtheorem{alg}{Algorithm}
% ========================================================


% Title Page
\title{Notes on Multi-target Tracking}
\author{Jakub Prüher}


\begin{document}

\maketitle
\begin{abstract}
	This report summarizes my understanding of the multi-target tracking (MTT) problems and the most representative algorithms. 
	First, I cover prerequisites necessary for understanding multi-target tracking such as basics of single-target tracking (Bayesian filtering), measurement formation and data association.
	The aim is to describe the most representative MTTs, such as joint probabilistic data association filter (JPDAF), multiple hypothesis tracking (MHT) and finally, my favourite, random finite set (RFS) approaches.
\end{abstract}




\chapter{Prerequisites}\label{ch:prerequisites}
\section{Single-Target Tracking}\label{sec:single-target_tracking}
\section{Data Association}\label{sec:data_association}
Global nearest neighbor (GNN)\\
All-neighbors data association - broader concept, PDA is a special case\\
Gating - 

\subsection{Probabilistic Data Association}
\subsection{Non-Bayesian Association Techniques}
\subsubsection{Nearest Neighbor}
selects the nearest measurement to the predicted measurement according to the normalized innovation squared (NIS) metric
Nearest Neighbor Standard Filter \citep{Bar-Shalom1995}, used in radar trackers

\subsubsection{Strongest Neighbor}
assuming that signal intensity information is available, selects the measurement with highest signal intensity
Strongest Neighbor Standard Filter \citep{Bar-Shalom1995}, used in sonar trackers

\subsubsection{Track-Split}
\subsection{Bayesian Association Techniques}
\subsubsection{Probabilistic Data Association}
\section{Single-Target Tracking in Clutter}\label{sec:single-target_tracking_in_clutter}

\chapter{Joint Probabilistic Data Association Filter}\label{ch:jpda_filter}
\chapter{Multiple Hypothesis Tracking}\label{ch:multiple_hypothesis_tracking}




\chapter{Random Finite Sets}\label{ch:random_finite_sets}

\begin{defn}[\emph{Set Integral}]
	 of a set function \( \pi \) is defined as
	\begin{equation}\label{eq:set_integral}
		\int_S \pi(X) \dsx[X] \triangleq \sum_{k=0}^{\infty}\frac{1}{k!}\int_{S^k} \pi(\Bqty{\inVar_1, \inVar_2, \ldots, \inVar_k}) \dx[(\inVar_1, \inVar_2, \ldots, \inVar_k)]
	\end{equation}
	with the convention \( \int_{S^0} \pi(\emptyset) \dsx[\emptyset] = \pi(\emptyset) \).
\end{defn}

\begin{defn}[\emph{FISST Density}]
	 is a set function \( \pi(X) \) such that 
	\begin{equation}\label{eq:fiist_density}
		\Pr(S \subseteq \mathcal{X}) = \int_S \pi(X) \dsx[X]
	\end{equation}
\end{defn}
FISST density \emph{is not} a probability density.

\begin{defn}[\emph{Cardinality Density}]
	of a RFS \( X \) is given by
	\begin{equation}\label{key}
		\rho(n) = \Pr(|X|=n) = \frac{1}{n!}\int \pi(\Bqty{\stVar_1,\, \ldots,\, \stVar_N}) \dx[\stVar_1]\cdot\ldots\cdot\dx[\stVar_N]
	\end{equation}
\end{defn}

\begin{defn}[\emph{I.I.D. Cluster RFS}]
	RFS with FISST density
	\begin{equation}\label{key}
		\pi(X) = n! \rho(n) \prod_{n=1}^{N} p(\stVar_n)
	\end{equation}
	containing elements that are i.i.d. according to a probability density \( p(\stVar): \stSp\to\R \).
\end{defn}

\begin{defn}[\emph{Poisson RFS}]
	I.I.D cluster with Poisson cardinality density, that is
	\begin{equation}\label{key}
		\rho(n \mid \lambda) = e^{-\lambda} \frac{\lambda^n}{n!}
	\end{equation}
	where \( \lambda \) is the expected number of objects.
\end{defn}

\begin{defn}[\emph{Bernoulli RFS}]
\begin{equation}\label{eq:bernoulli_rfs}
	\pi(X) = 
	\begin{cases}
	1 - r, 			& X = \emptyset \\
	r p(\stVar), 	& X = \Bqty{\stVar}
	\end{cases}
\end{equation}
\end{defn}

\begin{defn}[\emph{Multi-Bernoulli RFS}]
	Union of independent Bernoulli RFSs.
\end{defn}

\begin{defn}[\emph{Probability Hypothesis Density}]
	content...
\end{defn}
Probability hypothesis density (PHD) is also known as intensity function.

\begin{defn}[]
	content...
\end{defn}



\section{Probability Hypothesis Density Filter}



\bibliographystyle{plainnat}
\bibliography{refdb}
\end{document}          
