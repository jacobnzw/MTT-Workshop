\documentclass[a4paper]{scrreprt}
\usepackage[top=2cm,bottom=4cm]{geometry}
\usepackage[english]{babel}
\usepackage[T1]{fontenc}  			% make sure to install cm-super package for improved PDF quality
\usepackage[utf8]{inputenc}
\usepackage{lmodern}
%\usepackage[cm]{fullpage}			% Fullpage for now
\usepackage{graphicx}
\usepackage[dvipsnames]{xcolor}
\usepackage[font=small]{caption}
\usepackage{subcaption}
\usepackage{microtype}			 	% microtypographical fine-tunning
\usepackage{setspace} 				% Provides support for setting the spacing between lines in a document.
%\usepackage{tocbibind}				% Add bibliography/index/contents to Table of Contents.
\usepackage{mathtools}				% enhances amsmath (no need to load amsmath)
\usepackage{amssymb}
\usepackage{amsfonts}
\usepackage{amsthm}
\usepackage{bm} 					% bold math symbols including greek letters
\usepackage{booktabs}				% professional tables (w/o vertical bars)
\usepackage{array}					% 
\usepackage{dcolumn} 				% align to decimal point in tables
\usepackage[shortcuts]{extdash} 	% hyphenation of dashed words
%\usepackage[algoruled]{algorithm2e}
%\newcommand\mycommfont[1]{\small\ttfamily #1}
%\SetCommentSty{mycommfont}
\usepackage{cleveref}				% clever environment referencing
\crefname{figure}{Fig.}{Figs.}
\crefname{algorithm}{Alg.}{Algs.}
\usepackage{siunitx}				% typesetting of SI units
%\usepackage{paralist}		     	% compact lists with more options
%\usepackage[]{todonotes}			% TODO notes
%\presetkeys{todonotes}{backgroundcolor=orange!50}{}
% Not sure which to use for bibliography
%\usepackage{cite}
%\usepackage{citesort}
\usepackage[square, sort, numbers, authoryear]{natbib}
\usepackage{sty/macros}


% ======================== THEOREMS ========================
\theoremstyle{theorem}
\newtheorem{thm}{Theorem}
\newtheorem{lem}{Lemma}
\newtheorem{cor}{Corollary}
\theoremstyle{definition}
\newtheorem{defn}{Definition}
\newtheorem{alg}{Algorithm}
% ========================================================


% Title Page
\title{Notes on Multi-target Tracking}
\author{Jakub Prüher}


\begin{document}
\maketitle

\begin{abstract}
	This report summarizes my understanding of the multi-target tracking (MTT) problems and the most representative algorithms. 
	First, I cover prerequisites necessary for understanding multi-target tracking such as basics of single-target tracking (Bayesian filtering), measurement formation and data association.
	The aim is to describe the most representative MTTs, such as joint probabilistic data association filter (JPDAF), multiple hypothesis tracking (MHT) and finally, my favourite, random finite set (RFS) approaches.
\end{abstract}




\chapter{Prerequisites}\label{ch:prerequisites}


\section{Single-Target Tracking}\label{sec:single-target_tracking}


\section{Single-Target Tracking in Clutter}\label{sec:single-target_tracking_in_clutter}


\section{Data Association}\label{sec:data_association}




\chapter{Joint Probabilistic Data Association Filter}\label{ch:jpda_filter}




\chapter{Multiple Hypothesis Tracking}\label{ch:multiple_hypothesis_tracking}




\chapter{Random Finite Sets}\label{ch:random_finite_sets}

FIIST density is a function ... such that 
\begin{equation}\label{eq:fiist_density}
	\mathrm{Pr}(S \subseteq \mathcal{X}) = \int_S \pi(X) \dsx[X]
\end{equation}
where the \emph{set integral} is defined as
\begin{equation}\label{eq:set_integral}
	\int_S \pi(X) \dsx[X] \triangleq \sum_{k=0}^{\infty}\frac{1}{k!}\int_{S^k} \pi(\Bqty{\inVar_1, \inVar_2, \ldots, \inVar_k}) \dx[(\inVar_1, \inVar_2, \ldots, \inVar_k)]
\end{equation}
with the convention \( \int_{S^0} \pi(\emptyset) \dsx[\emptyset] = \pi(\emptyset) \).

FIIST density is not a probability density.

\begin{defn}[\emph{Bernoulli RFS}]
\begin{equation}\label{eq:bernoulli_rfs}
	\pi(X) = 
	\begin{cases}
	1 - r, 			& X = \emptyset \\
	r p(\stVar), 	& X = \Bqty{\stVar}
	\end{cases}
\end{equation}
\end{defn}

\begin{defn}[\emph{Multi-Bernoulli RFS}]
	Union of independent Bernoulli random finite sets.
\end{defn}

\begin{defn}[\emph{I.I.D. Cluster RFS}]
	RFS containing elements that are i.i.d. according to a probability density \( p(\stVar): \stSp\to\R \)
\end{defn}

\begin{defn}[\emph{Probability Hypothesis Density (Intensity Function)}]
	content...
\end{defn}

\begin{defn}[]
	content...
\end{defn}



\section{Probability Hypothesis Density Filter}




\end{document}          
